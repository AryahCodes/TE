\documentclass[margin=5pt]{standalone}
\usepackage{pgfplots}
\usepackage{tikz}
\usepgfplotslibrary{groupplots}
\pgfplotsset{compat=newest}

\usepackage{amsmath}
\usepackage{palatino}
\newlength{\mainwidth}
\setlength{\mainwidth}{460pt}



\pgfplotsset{colormap={inferno}{
    rgb(0)=(0.001462, 0.000466, 0.013866),
    rgb(15)=(0.037668, 0.025921, 0.132232),
    rgb(30)=(0.116656, 0.047574, 0.272321),
    rgb(45)=(0.217949, 0.036615, 0.383522),
    rgb(60)=(0.316282, 0.053490, 0.425116),
    rgb(75)=(0.410113, 0.087896, 0.433098),
    rgb(90)=(0.503493, 0.121575, 0.423356),
    rgb(105)=(0.596940, 0.154848, 0.398125),
    rgb(120)=(0.688653, 0.192239, 0.357603),
    rgb(135)=(0.775059, 0.239667, 0.303526),
    rgb(150)=(0.851384, 0.302260, 0.239636),
    rgb(165)=(0.912966, 0.381636, 0.169755),
    rgb(180)=(0.956852, 0.475356, 0.094695),
    rgb(195)=(0.981895, 0.579392, 0.026250),
    rgb(210)=(0.987464, 0.690366, 0.079990),
    rgb(225)=(0.973088, 0.805409, 0.216877),
    rgb(240)=(0.947594, 0.917399, 0.410665),
    rgb(255)=(0.988362, 0.998364, 0.644924),
    }} 

\pgfplotsset{
    colormap={magma_inverted}{
        rgb255=(252,253,191)
        rgb255=(254,194,135)
        rgb255=(251,135,97)
        rgb255=(229,80,100)
        rgb255=(181,54,122)
        rgb255=(129,37,129)
        rgb255=(79,18,123)
        rgb255=(28,16,68)
        rgb255=(0,0,4)
    }
}
\begin{document}

\begin{tikzpicture}

    \begin{axis}[
        name=plot1,
        width=6cm, height=6cm,
        title=Tower Chebyshev 4 Qubits,
        view={0}{90},
        colormap name= inferno,
        colorbar,
        xlabel=$t$,
        ylabel=$x$,
        ylabel shift = -10pt,
        point meta min=-1,
        point meta max=1,
        colorbar style= {
            ytick={-1,0,1},
            yticklabels={$-1$,$0$,$1$},
        }
    ]
    \addplot3[
        surf,
        shader=interp,
        mesh/cols=50,
        mesh/rows=50,
        mesh/ordering=rowwise,
    ] table [x=x, y=y, z=tower_chebyshev, col sep=space] {tower_cheb_comp.dat};
    \end{axis}

    \begin{axis}[
        name=plot2,
        at={(plot1.right of south east)}, 
        anchor=south west,
        xshift=2.5cm,
        width=6cm, height=6cm,
        title={$\left| u(t,x)_{\text{ref.}} - u_{\text{Tower Cheb.}}(t,x) \right|$},
        xlabel=$x$,
        ylabel=$y$,
        ylabel shift = -10pt,
        view={0}{90},
        colorbar,
        colormap name = magma_inverted,
    ]
    \addplot3[
        surf,
        shader=interp,
        mesh/cols=50,
        mesh/rows=50,
        mesh/ordering=rowwise,
    ] table [x=x, y=y, z expr = {abs(\thisrow{solution} - \thisrow{tower_chebyshev})}, col sep=space] {tower_cheb_comp.dat};
  
    \end{axis}

    \begin{axis}[
        name=plot3,
        at={(plot1.below south west)}, 
        anchor=north west,
        yshift=-1.5cm,
        width=6cm, height=6cm,
        title={TE-QPINN 4 Qubits},
        xlabel=$t$,
        ylabel=$x$,
        ylabel shift = -10 pt,
        view={0}{90},
        colormap name= inferno,
        colorbar,
        point meta min=-1,
        point meta max=1,
        colorbar style= {
            ytick={-1,0,1},
            yticklabels={$-1$,$0$,$1$},
        }
    ]
    \addplot3[
        surf,
        shader=interp,
        mesh/cols=50,
        mesh/rows=50,
        mesh/ordering=rowwise,
    ] table [x=x, y=y, z=fnn_basis, col sep=space] {fnn_basis_comp.dat};
    \end{axis}

    \begin{axis}[
        name=plot4,
        at={(plot3.right of south east)}, 
        anchor=south west,
        xshift=2.5cm,
        width=6cm, height=6cm,
        title={$\left| u(t,x)_{\text{ref.}} - u_{\text{TE-QPINN}}(t,x) \right|$},
        xlabel=$t$,
        ylabel=$x$,
        ylabel shift = -10 pt,
        view={0}{90},
        colormap name = magma_inverted,
        colorbar,
    ]
    \addplot3[
        surf,
        shader=interp,
        mesh/cols=50,
        mesh/rows=50,
        mesh/ordering=rowwise,
    ] table [x=x, y=y, z expr = {abs(\thisrow{solution} - \thisrow{fnn_basis})}, col sep=space] {fnn_basis_comp.dat};
    \end{axis}
    
\end{tikzpicture}


\end{document}