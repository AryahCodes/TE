\documentclass{article}
\usepackage{pgfplots}
\pgfplotsset{compat=1.18}
\usepackage{pgfplotstable}
\usepackage{palatino}
\usepackage{booktabs}
\usepackage{siunitx}
\usepackage{makecell}
\usepackage{multirow}
\usepackage{diagbox}


\usepgfplotslibrary{groupplots}
\newlength{\mainwidth}
\setlength{\mainwidth}{460pt}

\pgfplotsset{
 colormap={magma_inverted}{
 rgb255=(252,253,191)
 rgb255=(254,194,135)
 rgb255=(251,135,97)
 rgb255=(229,80,100)
 rgb255=(181,54,122)
 rgb255=(129,37,129)
 rgb255=(79,18,123)
 rgb255=(28,16,68)
 rgb255=(0,0,4)
 }
}

\begin{document}

\begin{tikzpicture}
\begin{axis}[
    group style={
        group size=2 by 1, % 2 plots side by side, 1 row
        horizontal sep=4cm, % Separation between the plots
    },
    width=0.45\textwidth, % Adjust width of each plot
    title={Loss},
    xlabel={Qubits},
    ylabel={Layers},
    xtick={2, 4, 6, 8},
    xticklabels={2,4,6,8},
    ytick={1, 3, 5, 7, 10},
    yticklabels={1,3,5,7,10},
    colorbar,
    colormap name= magma_inverted,
    enlargelimits=false,
]

% First plot
% \nextgroupplot
% \addplot [
%  matrix plot*,
%  mesh/cols=4,
%  mesh/rows=5,
%  point meta=explicit,
% ] table [x=N_WIRES, y=N_LAYERS, meta=loss] {data_tower_chebyshev.dat};

% Second plot
%\nextgroupplot
\addplot [
 matrix plot*,
 mesh/cols=4,
 mesh/rows=5,
 point meta=explicit,
 y dir=reverse
] table [x=N_WIRES, y=N_LAYERS, meta=loss] {data_te_qpinn.dat};

\end{axis}
\end{tikzpicture}

\begin{table}[h]
    \centering
    \sisetup{scientific-notation=true, round-mode=places, round-precision=2}
    \begin{tabular}{@{}c S[table-format=1.2e-1,mode=text] S[table-format=1.2e-1,mode=text] S[table-format=1.2e-1,mode=text] S[table-format=1.2e-1,mode=text]@{}}
    \toprule
    \multicolumn{1}{c}{\diagbox{Layers}{Wires}} & {2} & {4} & {6} & {8} \\
    \midrule
    1  & 1.27e+1 & 7.56e+0 & 7.05e+0 & 5.83e+0 \\
    3  & 5.16e+0 & 3.30e-1 & 2.83e-3 & 2.88e-3 \\
    5  & 5.16e+0 & 3.20e-1 & 6.38e-6 & 3.99e-6 \\
    7  & 5.16e+0 & 3.20e-1 & 2.12e-5 & 6.57e-6 \\
    10 & 5.16e+0 & 3.20e-1 & 9.54e-6 & 1.60e-5 \\
    \bottomrule
    \end{tabular}
    \caption{Loss values for different combinations of Layers and Wires using the Chebyshev quantum feature map}
    \label{tab:loss_values}
    \end{table}

    
    \begin{table}[h]
    \centering
    \sisetup{scientific-notation=true, round-mode=places, round-precision=2}
    \begin{tabular}{@{}c S[table-format=1.2e-1,mode=text] S[table-format=1.2e-1,mode=text] S[table-format=1.2e-1,mode=text] S[table-format=1.2e-1,mode=text]@{}}
    \toprule
    \multicolumn{1}{c}{\diagbox{Layers}{Wires}} & {2} & {4} & {6} & {8} \\
    \midrule
    1  & 7.33e-4 & 8.37e-5 & 1.20e-4 & 5.10e-4 \\
    3  & 8.63e-3 & 3.33e-5 & 4.66e-5 & 7.31e-6 \\
    5  & 3.00e-2 & 4.19e-5 & 1.03e-5 & 1.77e-5 \\
    7  & 2.77e-3 & 2.03e-5 & 1.75e-5 & 1.28e-5 \\
    10 & 2.83e-4 & 2.25e-5 & 7.59e-6 & 1.61e-5 \\
    \bottomrule
    \end{tabular}
    \caption{Loss values for different combinations of Layers and Wires using the TE-QPINN.}
    \label{tab:loss_values}
\end{table}
    
\end{document}
